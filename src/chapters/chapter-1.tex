\chapter{Pendahuluan}


\section{Latar Belakang}

Transport Layer Security (TLS) merupakan sebuah protokol yang banyak digunakan untuk menjamin keamanan dan kerahasiaan dari sebuah pesan yang dikirim melalui jaringan. Saat ini, TLS digunakan sebagai sebuah layer security dari berbagai protokol yang saat ini umum digunakan di Internet seperti Hyper-Text Transfer Protocol (HTTP), File Transfer Protocol (FTP), ataupun Simple Mail Transfer Protocol (SMTP). Mengingat posisi protokol ini sebagai sebuah layer yang berbeda dibandingkan protokol-protokol lain yang digunakan dalam perangkat lunak, TLS juga dapat digunakan dalam protokol lain yang mungkin dibuat pada masa yang akan datang.

Secara sederhana, sebuah koneksi TLS akan didahului dengan adanya TLS handshake untuk menentukan symmetric key yang akan digunakan oleh client dan server untuk berkomunikasi setelahnya. Sayangnya, waktu yang digunakan untuk TLS handshake masih cukup signifikan. Penggunaan TLS diatas HTTP dapat mengakibatkan waktu load pada sebuah website menjadi 1,2 hingga 3 kali lebih lambat (Naylor et al, 2014). Hal ini merupakan salah satu alasan kenapa saat ini TLS belum digunakan oleh seluruh layanan yang ada di internet.

Pada sebuah TLS handshake, client akan melakukan koneksi terhadap server untuk menandakan bahwa ia ingin membuat sebuah TLS session baru yang akan digunakan untuk pengiriman data antara client dan server. Server kemudian akan membalas koneksi tersebut dengan sebuah certificate berisi public key miliknya. Client kemudian akan membuat sebuah pre-master symmetric key yang akan dienkripsi dengan public key yang diterima. Server kemudian akan mendekripsi data yang ia terima untuk mendapat pre-master symmetric key; pre-master symmetric key akan dilakukan komputasi lebih jauh untuk mendapatkan symmetric key yang akan digunakan pada komunikasi antara server dan client.

Dari beberapa langkah yang dilakukan dalam sebuah TLS handshake, komputasi dekripsi algoritma Rivest–Shamir–Adleman (RSA) merupakan salah satu bagian dapat dioptimasi dengan relatif mudah.  Terdapat beberapa cara yang telah disarankan untuk meningkatkan kinerja TLS handshake seperti melakukan batching pada dekripsi RSA di server, melakukan rebalancing sehingga komputasi pada client lebih sulit daripada server, ataupun melakukan optimisasi pada algoritma RSA sehingga komputasi dapat dilakukan dengan lebih cepat.

Saat ini telah terdapat beberapa implementasi RSA dalam algoritma pemrograman paralel baik dalam paralel dalam Central Processing Unit (CPU) maupun dalam Graphics Processing Unit (GPU). Pembuatan algoritma RSA secara multiprocessing dalam GPU bukanlah hal yang baru dan dapat menghasilkan troughput hingga 45 kali dari implementasi RSA pada CPU (Zhang et al, 2014). Namun mengingat sebagian besar server pada data center tidak memilki GPU, implementasi RSA pada GPU tidak dapat digunakan untuk meningkatkan kinerja TLS handshake.

Dewasa ini, penggunaan CPU dengan jumlah core yang tinggi pada server merupakan hal yang lumrah; seri Xeon dari Intel telah menyediakan jumlah core hingga 22. Jumlah core yang tinggi merupakan salah satu alasan mengapa penggunaan algoritma multiprocessing pada dekripsi RSA dapat meningkatkan kinerja dari sebuah TLS handshake.


\section{Rumusan Masalah}

Kinerja dari TLS handshake hingga saat ini masih merupakan salah satu hal yang menghambat implementasi TLS secara luas. Salah satu cara untuk meningkatkan kinerja dari TLS handshake adalah dengan melakukan optimasi pada algoritma dekripsi RSA. Akan tetapi, optimisasi algoritma RSA menggunakan pemrograman paralel pada CPU bukan merupakan topik yang banyak dibahas pada penelitian-penelitian saat ini. Maka dari itu, tugas akhir ini akan berfokus pada pengembangan algoritma RSA dalam pemrograman paralel dan implementasinya pada TLS handshake. Dalam pengembangannya, terdapat berapa masalah yang akan menjadi perhatian utama dalam tugas akhir ini, yaitu:

\begin{enumerate}
  \item Algoritma paralel seperti apa yang akan diimplementasikan?
  \item Bagaimana cara mengintegrasikan algoritma paralel tersebut ke dalam TLS handshake?
  \item Apakah penggunaan algoritma paralel dapat meningkatkan kinerja dari TLS handshake?
\end{enumerate}

\section{Tujuan}

Tujuan dari tugas akhir ini adalah menemukan sebuah algoritma paralel RSA yang dapat diimplementasikan pada TLS handshake. Dengan adanya algoritma paralel tersebut, diharapkan bahwa kinerja dari TLS handshake akan meningkat dan penggunaan TLS pada internet akan semakin banyak digunakan.

\section{Batasan Masalah}

Tugas akhir ini hanya membahas mengenai optimisasi algoritma RSA pada implementasi TLS handshake. Dengan demikian, tugas akhir tidak memperhatikan hal-hal berikut:

\begin{enumerate}
  \item Protokol pertukaran kunci TLS handshake.
  \item Proses pemilihan dan/atau pembuatan cipher suite, TLS session, dan simmetric key pada TLS.
  \item Proses enkripsi pre-master symmetric key pada TLS client.
  \item Kompresi data.
\end{enumerate}

\section{Metodologi}

Pengerjaan tugas akhir ini akan melalui beberapa tahap, yaitu:
\begin{enumerate}
  \item Analisis Masalah

  Tugas akhir akan dimulai dengan dilakukannya analisis mengenai masalah terhadap implementasi TLS handshake yang ada pada saat ini. Pada tahap ini akan dipelajari juga mengenai solusi-solusi untuk meningkatkan kinerja dari TLS handshake yang sudah tersedia pada saat ini. Dari solusi yang sudah ada tersebut, penulis akan mencari penyebab solusi itu belum diimplementasikan pada TLS handshake dan mencari solusi baru yang mungkin untuk diimplementasikan.

  \item Perancangan Solusi dan Pengujian

  Perancangan solusi dibuat sebagai kerangka dasar atas solusi dari masalah yang telah ditemukan pada langkah sebelumnya. Pada tahap ini akan dirancang arsitektur solusi beserta cara untuk menerapkannya pada TLS handshake yang sudah ada. Selain itu, akan dibuat juga rencana pengujian terhadap implementasi solusi. Rencana pengujian akan dibuat sehingga percobaan dilakukan dalam batas masalah yang ada dan dapat dilakukan sesuai metode sains.

  \item Implementasi Solusi

  Pada tahap ini, penulis akan mengimplementasikan rancangan solusi yang sudah dibuat dalam sebuah bentuk perangkat lunak yang dapat digunakan dalam dunia sehari-hari. Hasil implementasi yang dibuat tentu ditujukan untuk menyelesaikan masalah yang ada.

  \item Pengujian dan Evaluasi

  Pengujian akan dilakukan berdasarkan rancangan pengujian telah ada. Pengujian dilakukan untuk mengetahui apakah solusi yang diberikan merupakan solusi terbaik dari masalah yang ada. Hasil dari pengujian akan digunakan sebagai data evaluasi dan penarikan kesimpulan.

\end{enumerate}


\section{Sistematika Laporan}

Penulisan tugas akhir ini akan terdiri dari lima bab yaitu: BAB I Pendahuluan, BAB II Studi Literatur, BAB III Analisis Solusi, BAB IV Rancangan, Implementasi dan Pengujian, BAB V Penutup.

Bab satu memaparkan mengenai latar belakang pembuatan tugas akhir ini, masalah utama yang dibahas, tujuan pembuatan tugas akhir, serta metodologi yang akan dilakukan selama pembuatan tugas akhir. Bab ini dibuat sebagai pendahuluan dan akan mencakup ringkasan dari sebagian besar hal yang dibahas pada tugas akhir.

Bab dua akan membahas mengenai teori dasar yang sudah ada mengenai protokol TLS, algoritma RSA, serta pengembangan dari algoritma RSA melalui pemrograman paralel yang sudah ada. Teori yang diambil akan bersumber dari literatur terpercaya serta dokumentasi resmi dari protokol TLS.

Bab tiga menggambarkan solusi yang digunakan untuk menyelesaikan masalah yang ada, kenapa digunakan solusi tersebut, serta hipotesa peningkatan kinerja dari TLS handshake setelah solusi yang dibuat diimplementasikan.

Bab empat memperlihatkan rancangan implementasi dari solusi, proses implementasi solusi tersebut, serta hasil pengujian dari implementasi tersebut pada kinerja TLS handshake.

Bab lima mendeskripsikan hasil kesimpulan dari solusi yang dibuat untuk menyelesaikan masalah serta saran yang dapat dilakukan untuk pengembangan dan perbaikan agar implementasi ini dapat menghasilkan hasil yang lebih baik.
